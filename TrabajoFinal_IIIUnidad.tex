\documentclass[12pt,letterpaper]{article}
\usepackage[utf8]{inputenc}
\usepackage[spanish]{babel}
\usepackage{graphicx}
\usepackage[left=2cm,right=2cm,top=2cm,bottom=2cm]{geometry}
\usepackage{graphicx} % figuras
\usepackage{subfigure} % subfiguras
\usepackage{float} % para usar [H]
\usepackage{amsmath}
\usepackage{txfonts}
\usepackage{stackrel} 
\usepackage{multirow}
\usepackage{enumerate} % enumerados
\renewcommand{\labelitemi}{$-$}
\renewcommand{\labelitemii}{$\cdot$}
\title{Caratula}
\begin{document}




\title{Caratula}

\begin{titlepage}
\begin{center}
\large{UNERSIDAD PRIVADA DE TACNA}\\
\vspace*{-0.025in}
\begin{figure}[htb]
\begin{center}
\includegraphics[width=8cm]{./IMG/logo}
\end{center}
\end{figure}
\vspace*{0.15in}
INGENIERIA DE SISTEMAS  \\

\vspace*{0.5in}
\begin{large}
TITULO:\\
\end{large}

\vspace*{0.1in}
\begin{Large}
\textbf{Guía de Seguridad de Base de Datos} \\
\end{Large}

\vspace*{0.3in}
\begin{Large}
\textbf{CURSO:} \\
\end{Large}

\vspace*{0.1in}
\begin{large}
BASE DE DATOS II\\
\end{large}

\vspace*{0.3in}
\begin{Large}
\textbf{DOCENTE(ING):} \\
\end{Large}

\vspace*{0.1in}
\begin{large}
 Patrick Cuadros Quiroga\\
\end{large}

\vspace*{0.2in}
\vspace*{0.1in}
\begin{large}
\begin{flushleft}
Integrantes: \\
Fanny Luz Clemente Cruz \\
Acevedo Vásquez, Leonardo Fernando 	(2014047512) \\
Andía Bernedo, Josei Jomar 			(2014049093) \\
Condori Velarde, Sonia          	(2014049546) \\
Clemente Cruz, Fanny Luz    		(2014049550) \\
Flores Colque, Gisela           	(2014049547) \\
Llatasi Cohaila, Cristian Omar		(2014037546) \\
Morales Anquise, Tommy Edwards 		(2015050480) \\
Ticona Arcaya, Sergio Alexis		(2014049171) \\
Tapia Ticona, Lupe Carolina			(2014049548) \\
\end{flushleft}


\end{large}
\end{center}

\end{titlepage}




 \tableofcontents
 \newpage

 
\section{INTRODUCCIÓN} 
En el presente informe Se explicará cómo es que se debe realizar un respaldo de la información, en este caso el respaldo de una base de datos en Oracle 11g Enterprise Edition para el uso del asistente grafico para copias de seguridad (Enterprise Manager).
Además, se utilizará SQLDEVElOPER.exe para para conectar un usuario, también sirve para migración de bases de datos de MySQL a Oracle.\\ \\
Se explicará qué tipos de backups se pueden realizar en Oracle, algunas recomendaciones de cuando realizar las copias de seguridad además de copias de seguridad en modo consola y de manera grafica. \\ \\
Una copia de los datos que se puede utilizar para restaurar y recuperar los datos se denomina copia de seguridad. Las copias de seguridad le permiten restaurar los datos después de un error. Con las copias de seguridad correctas, puede recuperarse de multitud de errores como:\\  \\
- Errores de medios.\\
- Errores de usuario, por ejemplo, quitar una tabla por error.\\
- Errores de hardware, por ejemplo, una unidad de disco dañada o la pérdida Permanente de un servidor.\\
- Desastres naturales.\\
\\
Además, las copias de seguridad de una base de datos son útiles para fines administrativos habituales, como copiar una base de datos de un servidor a otro, configurar la creación de reflejo de la base de datos y el archivo, etc.
Para impedir la perdida de datos se debe disponer de una estrategia de copia de seguridad, hacer copias de seguridad con regularidad, también se debe considerar los tipos de respaldo que soporta Oracle, los respaldo y recuperación y su procedimiento(el plan, que debe incluir, medios de soporte a utilizar, cuando realizarlo, periodicidad) y las herramientas(Dónde guardarlos - distancia y accesibilidad, Quienes realizan y manejan los  respaldos?, Verificación del respaldo, Registro, consejos para realizar los respaldos e instalaciones grandes).


 \newpage
\section{Objetivos} 
\subsection{Generales}
- Desarrollar una Guía Técnica de estrategia de copias de Seguridad y Recuperación de Bases de Datos
\subsection{Especificos}
- Definir qué tipo de backup aplicar y en qué consiste cada uno.\\
- Explicar el impacto de las estrategias de backups en las necesidades del espacio\\

\newpage
\section{Marco Teorico}
\subsection{Copias de Seguridad y Restauracion de Base de Datos}
Una copia de los datos que se puede utilizar para restaurar y recuperar los datos se denomina copia de seguridad. Las copias de seguridad le permiten restaurar los datos después de un error. Con las copias de seguridad correctas, puede recuperarse de multitud de errores, por ejemplo:  \\
- Errores de medios. \\ 
- Errores de usuario, por ejemplo, quitar una tabla por error. \\
- Errores de hardware, por ejemplo, una unidad de disco dañada o la pérdida \\
- permanente de un servidor. \\
- Desastres naturales. \\

Ademas, las copias de seguridad de una base de datos son útiles para fines administrativos habituales, como copiar una base de datos de un servidor a otro, configurar la creación de reflejo de la base de datos y el archivo, etc. \\

\subsection{Como Impedir la Perdida de los Datos}

Impedir la perdida de datos es uno de los problemas más importantes que afrontan los administradores de sistemas.  \\
a)Disponer de una estrategia de copia de seguridad \\
 
Debe tener una estrategia de copia de seguridad para aminorar la pérdida de datos y recuperar los datos perdidos. 
Los datos se pueden perder como consecuencia de errores de hardware o de software, o bien por: \\ 
- El uso accidental o malintencionado de una instruccion DELETE. \\
- El uso accidental o malintencionado de una instruccion UPDATE; por ejemplo, no utilizar la clausula WHERE con una instruccion UPDATE (se actualizan todas las filas en lugar de una fila concreta de la tabla). \\
- Virus destructivos. \\
- Desastres naturales, como incendios, inundaciones y terremotos. \\
- Robo.  \\
Si utiliza una estrategia de copia de seguridad adecuada, puede restaurar los datos con un costo minimo sobre la produccion y reducir la posibilidad de que los datos se pierdan definitivamente. Piense en la estrategia de copia de seguridad como un seguro. Su estrategia de copia de seguridad debe devolver el sistema al punto en el que se encontraba antes del problema. Al igual que con una póliza de seguros, preguntese: ¿cuanto estoy dispuesto a pagar y cuantas perdidas puedo permitirme?.  \\

Los costos asociados con la estrategia de copia de seguridad incluyen la cantidad de tiempo que se emplea en diseñar, implementar, automatizar y probar los procedimientos de copia de seguridad. Aunque la perdida de datos no se puede impedir completamente, debe diseñar una estrategia de copia de seguridad para reducir el alcance de los daños. Cuando diseñe una estrategia de copia de seguridad, considere la cantidad de tiempo que se puede permitir que el sistema esté parado, así como la cantidad de datos que se puede admitir perder (si puede perderse alguno) en el caso de un error del sistema. \\

b) Hacer copias de seguridad con regularidad \\

La frecuencia con que haga las copias de seguridad de la base de datos depende de la cantidad de datos que esté dispuesto a perder y la actividad de la base de datos. Cuando haga copias de seguridad de bases de datos de usuario, tenga en cuenta los siguientes hechos e instrucciones: \\

- Puede hacer copias de seguridad de la base de datos con frecuencia si el sistema se encuentra en un entorno de proceso de transacciones en línea (OLTP, Online Transaction Processing). \\
- Puede hacer copias de seguridad de la base de datos con menos frecuencia si el sistema tiene poca actividad o se utiliza, principalmente, para la toma de decisiones. \\
- Debe programar las copias de seguridad cuando no se estén efectuando muchas actualizaciones en SQL Server. \\
- Tras determinar la estrategia de realización de copias de seguridad, puede automatizar el proceso con el Asistente para planes de mantenimiento de bases de datos. \\

\subsection{Tipos de Respaldo que Soporta Oracle}
- Completo: Se respalda toda la base de datos. \\
- Incremental: Debe tener previamente un respaldo completo. Respalda a medida que se realizan cambios. \\
- Diferencial: Debe tener previamente un respaldo completo. Respalda las diferencias existentes entre un respaldo y otro. \\
- Flashbacks: Permite de manera rápida volver a un estado anterior de la base de datos. \\
a) Respaldo y Recuperacion \\

Para determinar cuando hacer un respaldo, pensar de la siguiente manera: hacer una copia de respaldo justo antes del momento en que regenerar los datos ocasione mayor esfuerzo que hacer el respaldo.\\
El respaldo y recuperacion de datos es la generacion de una copia, en un momento determinado, de los datos del sistema, con vistas a su eventual reposición en caso de perdida. \\

El respaldo y recuperacion de informacion, trata del esfuerzo necesario para asegurar la continuidad del procesamiento de los datos de las base de datos, con la mínima dificultad posible ante una eventual alteración no deseada de los mismos. \\

b)Respaldos \\

Respaldo es la obtención de una copia de los datos en otro medio magnético, de tal modo que a partir de dicha copia es posible restaurar el sistema al momento de haber realizado el respaldo. Por lo tanto, los respaldos deben hacerse con regularidad, con la frecuencia preestablecida y de la manera indicada, a efectos de hacerlos correctamente. \\

Es fundamental hacer bien los respaldos. De nada sirven respaldos mal hechos (por ejemplo, incompletos). En realidad, es peor disponer de respaldos no confiables que carecer totalmente de ellos. Suele ocurrir que la realización de respaldos es una tarea relegada a un plano secundario, cuando en realidad la continuidad de una aplicación depende de los mismos. Los respaldos son tan importantes como lo es el correcto ingreso de datos. \\

b.1) Procedimiento de Respaldo y Recuperación \\
 
- Aspectos a considerar \\
Dado que las aplicaciones (sistemas) tienen características inherentes, para cada aplicación
Corresponde un método apropiado de respaldo y recuperación de datos. Preferentemente, debe ser establecido por quienes desarrollan la aplicación, que son los que saben cuáles datos es necesario respaldar, la mejor manera de hacerlo, etc. y cómo hacer la correspondiente recuperación. Hay que tener en cuenta las características propias del usuario y cuál es la instalación en que funciona el sistema. Es decir, qué computadora, donde está instalada, etc. Incluye el área física (por ejemplo: ambiente aislado o transitado, acondicionamiento térmico, nivel de ruido, etc.).  \\
Algunos de los aspectos a considerar se presentan a continuación. La lista no es taxativa y el orden de cada aspecto no es relevante, siendo cada aspecto de propósito limitado en forma individual. Un adecuado método de respaldo/recuperación debe tener en cuenta todos los aspectos en conjunto, como ser:  \\
- Plan de Respaldo. \\
- Cuáles datos se deben incluir  \\
- Tipos de respaldos.  \\
- Cantidad de copias a realizar \\
- Modalidad de copia. \\
- Dónde guardarlas. \\
- Quienes los manejan. \\
- Verificación del respaldo. \\
- Registro. \\
- Cuándo hacerlo. \\
- El respaldo completo del disco. \\
- Soporte físico a utilizar para el respaldo  \\


-Plan de Respaldo \\ 
Los procedimientos de respaldo y recuperacion desarrollada deben formar parte de un plan de respaldo y recuperación, el cual debe ser documentado y comunicado a todas las personas involucradas. Dado que, a lo largo del tiempo, varias características que se consideran para desarrollar este plan sufren cambios (software utilizado, soporte,
etc.), el plan debe ser revisado, y de ser necesario modificado de manera periódica. 
 
El plan debe contener todos los ítems detallados a continuación y cualquier otro que mejore la realización del trabajo o clarifique la tarea. \\\\


-Cuáles datos se deben incluir? 
 
Cada aplicación maneja un conjunto muy variado de datos, algunos estáticos, otros dinámicos. Hay datos base, a partir de los cuales se generan datos resultantes (información). Al definir el respaldo, se establece si se copian todos los archivos o parte de ellos. Entra en consideración si la copia incluye los propios programas de la aplicación. La decisión final se tomará sobre la base de la criticidad de los datos y el valor de los mismos.  \\\\


-Medios de soporte a utilizar 
 Los medios a utilizar dependerán del tipo de computadora (micro. Mainframe, etc.), cantidad de información a almacenar, tiempo disponible para realizar el respaldo, costos y obviamente de la tecnología disponible al momento. \\
 Actualmente se cuenta con una variedad muy amplia de soportes disponibles y a costos muy bajos, sobre todo al nivel de micro computadoras. A nivel mainframe las posibilidades se acotan. Las características principales a considerar de cada opción será la capacidad de almacenamiento, medida en Mg o Gb, y la velocidad de transferencia de datos (cantidad de datos por segundo que es posible grabar), medida en Kb/s. Los soportes magnéticos son los más difundidos; también los hay ópticos.  \\\\

Las opciones más comunes son (en orden descendente por capacidad de almacenamiento promedio): \\ 
1.-Discos duros.\\
2.-Cintas (4mm, 8mm, QIC).\\
3.-Cartuchos (alta densidad, micro)\\
4.-C D R O M.\\
5.-Zip drive.\\
6.-Tarjetas PCM-CIA.\\
7.-Diskette  \\\\

\newpage

-¿Cuándo realizarlo? 

Para determinar cuándo se realiza el respaldo debemos conocer los tiempos incurridos en desarrollar la tarea y las ventanas de tiempo disponibles en producción (tiempo disponible para realizar tareas que no afecte a los procesos habituales de procesamiento de datos). Los tiempos incurridos en desarrollar la tarea varían dependiendo del soporte utilizado (a mayor velocidad de transferencia menor tiempo incurrido), el tipo de respaldo utilizado (el full back-up es el que lleva mayor cantidad de tiempo) y la cantidad de datos a respaldar (a mayor cantidad mayor tiempo de respaldo). 
Generalmente, en las empresas, las ventanas de tiempo disponibles durante las semanas son chicas, por lo que se combinan los tipos de respaldo  dependiendo del día de la semana. Así, el fin de semana se realiza un respaldo global y durante la semana se realizan respaldos incrementales. Existen momentos en los cuales es necesario realizar respaldos extraordinarios, como ser nueva instalación de una aplicación, o migración de bases de datos o traslado del equipamiento. En estas circunstancias deben realizarse un respaldo global, para estar totalmente cubierto. \\







\subsection{Marco Conceptual}

\newpage

\section{Desarrollo}
\subsection{Procedimiento para la Creacion de Copias}

\begin{enumerate}
\bf\underline{RECUPERACIÓN  DESDE  ENTERPRISE  MANAGER:}
\end{enumerate}
Para realizar una recuperación desde EM, iremos a “Disponibilidad” y seleccionamos Realizar Recuperación: \\
\\
\includegraphics[width=14cm]{./IMG/img40.jpg}

En ámbitos de recuperación podemos seleccionar toda o parte de la base de datos para recuperar. \\
\\
Para  el  ejemplo  hemos  borrado  el  datafile  USERS01.DBF(OFFLINE)  después  de realizar el backup y ahora vamos a intentar recuperarlo. Para ello usaremos la copia que acabamos de realizar. \\
\\
Iniciamos oracle en modo mount y arrancamos EM. \\
\\
Al no poder iniciar nos encontramos con esto una vez logueados.

\includegraphics[width=14cm]{./IMG/img41.jpg}

Pinchamos en Realizar Recuperación. Introducimos los credenciales de host. Continuar Nos conectamos como sysdba.

\newpage
\includegraphics[width=14cm]{./IMG/img42.jpg}

En el ámbito de recuperación elegimos Archivos de Datos y en el tipo de operación restaurar hasta hora actual. Pinchamos en recuperar.

\newpage
\includegraphics[width=14cm]{./IMG/img43.jpg}

Vemos  como  EM  localiza  la  ruta  en  conflicto  y  te  la  presenta  para  seleccionarla. Siguiente.


\newpage
\section{Referencias}
\newpage 
\section{Conclusiones} 

\end{document}
