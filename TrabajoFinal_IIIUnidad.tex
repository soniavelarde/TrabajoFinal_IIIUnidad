\documentclass[12pt,letterpaper]{article}
\usepackage[utf8]{inputenc}
\usepackage[spanish]{babel}
\usepackage{graphicx}
\usepackage[left=2cm,right=2cm,top=2cm,bottom=2cm]{geometry}
\usepackage{graphicx} % figuras
\usepackage{subfigure} % subfiguras
\usepackage{float} % para usar [H]
\usepackage{amsmath}
\usepackage{txfonts}
\usepackage{stackrel} 
\usepackage{multirow}
\usepackage{enumerate} % enumerados
\renewcommand{\labelitemi}{$-$}
\renewcommand{\labelitemii}{$\cdot$}
\author{Fanny Clemente}
\title{Caratula}
\begin{document}



\author{Fanny Clemente}
\title{Caratula}

\begin{titlepage}
\begin{center}
\large{UNERSIDAD PRIVADA DE TACNA}\\
\vspace*{-0.025in}
\begin{figure}[htb]
\begin{center}
\includegraphics[width=8cm]{./IMG/logo}
\end{center}
\end{figure}
\vspace*{0.15in}
INGENIERIA DE SISTEMAS  \\

\vspace*{0.5in}
\begin{large}
TITULO:\\
\end{large}

\vspace*{0.1in}
\begin{Large}
\textbf{Guía de Seguridad de Base de Datos} \\
\end{Large}

\vspace*{0.3in}
\begin{Large}
\textbf{CURSO:} \\
\end{Large}

\vspace*{0.1in}
\begin{large}
BASE DE DATOS II\\
\end{large}

\vspace*{0.3in}
\begin{Large}
\textbf{DOCENTE(ING):} \\
\end{Large}

\vspace*{0.1in}
\begin{large}
 Patrick Cuadros Quiroga\\
\end{large}

\vspace*{0.2in}
\vspace*{0.1in}
\begin{large}
\begin{flushleft}
Integrantes: \\
Fanny Luz Clemente Cruz \\
Acevedo Vásquez, Leonardo Fernando 	(2014047512) \\
Andía Bernedo, Josei Jomar 			(2014049093) \\
Condori Velarde, Sonia          	(2014049546) \\
Clemente Cruz, Fanny Luz    		(2014049550) \\
Flores Colque, Gisela           	(2014049547) \\
Llatasi Cohaila, Cristian Omar		(2014037546) \\
Morales Anquise, Tommy Edwards 		(2015050480) \\
Ticona Arcaya, Sergio Alexis		(2014049171) \\
Tapia Ticona, Lupe Carolina			(2014049548) \\
\end{flushleft}


\end{large}
\end{center}

\end{titlepage}




 \tableofcontents
 \newpage

 
\section{INTRODUCCIÓN} 
En el presente informe Se explicará cómo es que se debe realizar un respaldo de la información, en este caso el respaldo de una base de datos en Oracle 11g Enterprise Edition para el uso del asistente grafico para copias de seguridad (Enterprise Manager).
Además, se utilizará SQLDEVElOPER.exe para para conectar un usuario, también sirve para migración de bases de datos de MySQL a Oracle.\\ \\
Se explicará qué tipos de backups se pueden realizar en Oracle, algunas recomendaciones de cuando realizar las copias de seguridad además de copias de seguridad en modo consola y de manera grafica. \\ \\
Una copia de los datos que se puede utilizar para restaurar y recuperar los datos se denomina copia de seguridad. Las copias de seguridad le permiten restaurar los datos después de un error. Con las copias de seguridad correctas, puede recuperarse de multitud de errores como:\\  \\
- Errores de medios.\\
- Errores de usuario, por ejemplo, quitar una tabla por error.\\
- Errores de hardware, por ejemplo, una unidad de disco dañada o la pérdida Permanente de un servidor.\\
- Desastres naturales.\\
\\
Además, las copias de seguridad de una base de datos son útiles para fines administrativos habituales, como copiar una base de datos de un servidor a otro, configurar la creación de reflejo de la base de datos y el archivo, etc.
Para impedir la perdida de datos se debe disponer de una estrategia de copia de seguridad, hacer copias de seguridad con regularidad, también se debe considerar los tipos de respaldo que soporta Oracle, los respaldo y recuperación y su procedimiento(el plan, que debe incluir, medios de soporte a utilizar, cuando realizarlo, periodicidad) y las herramientas(Dónde guardarlos - distancia y accesibilidad, Quienes realizan y manejan los  respaldos?, Verificación del respaldo, Registro, consejos para realizar los respaldos e instalaciones grandes).


 \newpage
\section{Objetivos} 
\subsection{Generales}
- Desarrollar una Guía Técnica de estrategia de copias de Seguridad y Recuperación de Bases de Datos
\subsection{Especificos}
- Definir qué tipo de backup aplicar y en qué consiste cada uno.\\
- Explicar el impacto de las estrategias de backups en las necesidades del espacio\\

\newpage
\section{Marco Teorico}
\subsection{Copias de Seguridad y Restauracion de Base de Datos}


\subsection{Como Impedir la Perdida de los Datos}

\subsection{Tipos de Respaldo que Soporta Oracle}
\subsection{Marco Conceptual}

\newpage

\section{Desarrollo}
\subsection{Procedimiento para la Creacion de Copias}

\newpage
\section{Referencias}
\newpage 
\section{Conclusiones} 

\end{document}
